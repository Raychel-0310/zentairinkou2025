\documentclass{article}
\usepackage{titlesec}
\usepackage[a4paper,margin=2.5cm]{geometry}
\usepackage{lmodern}
\usepackage{amsmath}

\begin{document}

\title{第1回輪講資料 『徹底解説 電動機・発電機の理論』 pp.55-73}
\author{著者: 山口正人・横関政洋 \\ 担当: 脇本 怜奈}
\date{\today}
\maketitle

\section*{概要}


\setcounter{section}{3}
\setcounter{equation}{40}
\renewcommand{\thesection}{2.\arabic{section}}
\renewcommand{\theequation}{2.\arabic{equation}}

\section{電力工学のための数学的道具} 

\subsection{オイラーの公式とフェーザ表示}
まず、\(\theta\) を実数として、オイラーの公式を式(2.41)に導入する。
\begin{equation}
    e^{j\theta} = \cos\theta + j \sin\theta
\end{equation}
ここで、\(\theta\)を0から2\(\pi\)まで変化させたとき、(2.41)式は\(\theta\)を偏角として半径1の単位円を描く周期関数である。

任意の複素数は、式(2.42)のように\(e^{j\theta}\)を用いて表すことが可能で、
\begin{equation}
    z = re^{j\theta}
\end{equation}
である。
(2.41)式を用いて(2.42)式を展開すれば、
\begin{align}
    z &= re^{j\theta} \\
       &=r(\cos\theta + j\sin\theta) \\
       &=r\cos\theta + jr\sin\theta
\end{align}
\(a=r\cos\theta, b=r\sin\theta\)とすると、図形的に以下の図のように表すことが可能となる。

\end{document}
